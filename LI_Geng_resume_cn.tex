%%%%%%%%%%%%%%%%%%%%%%%%%%%%%%%%%%%%%%%%%
% Medium Length Professional CV
% LaTeX Template
%
% This template has been downloaded from:
% http://www.LaTeXTemplates.com
%
% Original author:
% Trey Hunner (http://www.treyhunner.com/)
%
% Important note:
% This template requires the resume.cls file to be in the same directory as the
% .tex file. The resume.cls file provides the resume style used for structuring the
% document.
%
%%%%%%%%%%%%%%%%%%%%%%%%%%%%%%%%%%%%%%%%%

%----------------------------------------------------------------------------------------
%	PACKAGES AND OTHER DOCUMENT CONFIGURATIONS
%----------------------------------------------------------------------------------------

\documentclass{resume} % Use the custom resume.cls style
\usepackage{CJKutf8}
\usepackage[left=0.75in,top=0.6in,right=0.75in,bottom=0.6in]{geometry} % Document margins

\begin{CJK}{UTF8}{gbsn}
\name{李庚} % Your name
\address{https://stackoverflow.com/users/486149/angus-lee} % Personal websites
\address{(+86) 13xxxxxxxxx \\ angus.lee8329@gmail.com} % Phone number and email

\begin{document}

%----------------------------------------------------------------------------------------
%	SUMMARY SECTION
%----------------------------------------------------------------------------------------
\begin{rSection}{个人简介}
  \item 熟悉 Web 后端研发,Linux 操作系统级别的性能分析与容量规划, JVM 性能调优;熟悉 C/C++, Java, Common Lisp, Scala, Python 等程序设计语言;
  \item 熟悉技术团队管理;
  \item 熟练掌握英语。
\end{rSection}

%----------------------------------------------------------------------------------------
%	TECHNICAL STRENGTHS SECTION
%----------------------------------------------------------------------------------------

%\begin{rSection}{Technical Strengths}

%\begin{tabular}{ @{} >{\bfseries}l @{\hspace{6ex}} l }
%System Analysis and Architect \\
%Programming Languages & C, C++, Java, Lisp \\
%Operating Systems & Linux \\
%\end{tabular}

%\end{rSection}

%----------------------------------------------------------------------------------------
%	WORK EXPERIENCE SECTION
%----------------------------------------------------------------------------------------

\begin{rSection}{职业经历}

  \begin{rSubsection}{OpenResty.com, Inc}{2020/07 - 2021/09}{软件研发工程师}{北京}
  \item OpenResty 产品研发:
    \begin {itemize}
    \item 支持在 HTTPS 连接下在 Intel QAT 硬件中完成 SSL 加解密运算;
    \item 集成 Nginx QUIC (HTTP3) 特性。
    \end{itemize}
  \item 时序数据库的设计与管理维护。
  \end{rSubsection}

  \begin{rSubsection}{去哪儿网 (www.qunar.com)}{2010/07 - 2020/06}{软件研发工程师}{北京}

  \item 旅游度假事业部技术总监
    \begin{itemize}
    \item 旅游度假搜索服务架构设计与开发运维;
    \item 旅游度假竞价排名系统架构设计与开发运维;
    \item {
        Linux操作系统和Java虚拟机层面的持续性能优化:
        \begin{itemize}
        \item { 消除搜索引擎服务对网络带宽的过度使用,使搜索引擎服务的P98响应时间指标降低60\%; }
        \item { 消除内存泄漏导致的 CPU 过度占用,提升了整体服务性能且降低了响应时间的波动; }
        \item { JVM性能评估与调优:JDK 8 升级到 JDK 11,提升服务性能约15\% }
        \end{itemize}
      }
    \item 旅游度假技术团队的开发流程规范制定。
    \end{itemize}
  \item 技术委员会成员
    \begin{itemize}
    \item 制定全公司级别 Java 和 Python 两个技术线的统一职级晋升评审标准;
    \item 规范全公司层面的 Java 代码的书写标准以及风险代码提交的管理流程;
    \item 负责全公司层面高级别技术晋升答辩评审。
    \end{itemize}
    
  \end{rSubsection}

  % ------------------------------------------------

  \begin{rSubsection}{百度公司 (www.baidu.com)}{2009/10 - 2009/12}{实习软件工程师}{北京}
  \item 参与大数据存储计算平台系统研发团队
  \end{rSubsection}

\end{rSection}

% ----------------------------------------------------------------------------------------
%	PROJECTS SECTION
% ----------------------------------------------------------------------------------------

% ----------------------------------------------------------------------------------------
%	EDUCATION SECTION
% ----------------------------------------------------------------------------------------

\begin{rSection}{教育背景}

  {\bf 北京大学} \hfill {\em 2010/07} \\ 
  信息科学技术学院计算机系统结构专业,硕士 \\
  北京奥组委国家会议中心击剑馆实习生

  {\bf 北京大学} \hfill {\em 2006/07} \\ 
  主修计算机科学与技术专业,学士 \\
  辅修法语

\end{rSection}


% ----------------------------------------------------------------------------------------
%	EXAMPLE SECTION
% ----------------------------------------------------------------------------------------

% \begin{rSection}{Section Name}

%   Section content\ldots

% \end{rSection}

% ----------------------------------------------------------------------------------------

\end{CJK}
\end{document}
